\documentclass[a4paper,12pt]{article}
\usepackage{graphicx}
\usepackage[top=2cm,bottom=2.5cm,left=2cm,right=2cm]{geometry}
\usepackage{xeCJK}
\usepackage{indentfirst}
\usepackage{color}
\usepackage{epsfig}
\setlength{\baselineskip}{10pt}
\setlength{\parindent}{2em}

\makeatletter
\renewcommand\section{\@startsection{section}{0}{\z@}%
{-2.5ex \@plus -1ex \@minus -.2ex}%
{1.3ex \@plus.2ex}%
{\normalfont\large\CJKfamily{hei}}}

\renewcommand\subsection{\@startsection{subsection}{1}{\z@}%
{-1.5ex \@plus -1ex \@minus -.2ex}%
{0.5ex \@plus.2ex}%
{\normalfont\normalsize\CJKfamily{hei}}}
\makeatother

\begin{document}
%\begin{CJK*}{GBK}{kai}
\fontsize{12}{20pt}\selectfont

\title{观测策略使用手册 $_{1.0}$}
\author{郑捷}
\date{2016年2月}

\maketitle

%\begin{center}
%{\CJKfamily{hei}
%{\Large 观测策略使用手册 $_{1.0}$}\\
%{郑捷}\\
%{2016年2月}
%}
%\end{center}

%\setcounter{section}{-1}

\section*{概述}

本策略根据现行观测实际需要情况,以及未来观测计划需要,进行编写。

\section{运行环境}

本策略程序在IDL环境中运行,要求操作系统为Linux或Mac OS。

如果在Windows下运行,需要将其中 set\_plot, `x' ,改为 set\_plot, `win' 。

\section{程序以及文件介绍}

\subsection{目录分配}

在程序根目录中,直接存放 *.pro 等程序代码,执行也完全在本目录下进行。操作全部采用相对路径,
可以在任何工作目录下存放和运行。

conf 目录存放系统配置文件,包括观测策略等。

obsed 目录存放已经观测过的数据的汇总情况。

plan 目录存放生成的观测计划,以及生成计划时的相关文件。

\subsection{策略主程序模块}

主程序为 z\_Planner ,以及 z\_Planner\_load ,后者为被包含文件,主要是为了便于阅读。
主程序还调用了 zh\_alt2air.pro 、 zh\_plotairmass.pro 、 zj\_chooseblock.pro 、 
zj\_loadunobsed.pro 、 zj\_radec2str.pro 等模块,这些模块作为pro/function等提供必要的功能支持。
主程序还调用了一些自编的库文件,以及部分简单的转换函数等,此处不额外列出。

\subsection{观测结果汇总模块}

模块 z\_Check 和 z\_Collect 负责对已经观测得到的文件进行检查,生成已观测列表等。
其中调用了 z\_HeaderInfo 模块,对fits文件头进行解析。此处将HeaderInfo模块独立,是为了适应
不同望远镜的观测文件头的不同,将策略应用到其他望远镜时,只需要修改本文件即可。

\subsection{视场和观测区块划分模块}

使用 z\_make\_field 、z\_make\_block 进行视场和天区划分。对于目前现有的天区划分,使用
z\_import\_block 进行导入,不再重新划分,确保和历史数据的一致性。

\subsection{配置文件 conf/}

\paragraph{basic.txt} 本文件主要包括观测站点信息等,每行为一个信息,依次为:\\
观测站经度,三个值,度、分、秒,用空格隔开,之后是注释,注释不会被读取(下同)\\
观测站纬度,度、分、秒\\
观测站海拔高度,米为单位\\
观测站所在时区,-12到12\\
曝光时间之外的每次观测额外时间,秒,包括CCD读出时间、望远镜指向时间等,
该时间为估算平均时间,不是真实间隔,需要根据经验设置。\\
望远镜视场大小,度,用于绘图,大概估计即可,不影响策略本身。

在第一行之前不能有空行,在各行数据之间也不能有空行,否则会读出错误。

\paragraph{exp\_plan.txt} 曝光计划文件,描述对每个天区的曝光计划,每行为一次曝光,
各列分别为:计划代码(从0开始顺序编号)、滤光片、曝光时间、重复次数、单次曝光的完成率、
是否dither(0或1)、曝光方案名称。

重复次数和完成率,主要针对多次短曝光组合成的长曝光进行设置,例如南山一米望远镜用2次150s
曝光来代替单次300s,则此处设置为 150  2  0.5 ,表示每次曝光150s,共需要2次,每次完成
该任务的0.5 。在策略中,如果检测到部分完成的曝光,会安排适当的次数进行补全。

\paragraph{exp\_factor.txt} 曝光因子文件,描述实际每次曝光(滤光片+时间),对应上面
的哪一个计划,以及相应的完成率。

例如南山的150s和300s,都对应300s长曝光方案,则在因子文件中要出现两次,分别给予不同的完成率。

\paragraph{dither.txt} Dither配置,在第一行用空格隔开两个浮点数,分别表示每次dither
时在RA和Dec方向上的偏移量。注意,只能写在第一行,之前不能有空行。之后的内容不会读入。

\paragraph{field.txt和block.txt} 视场划分和天区划分文件,定义视场编号、坐标、所属天区(条带)
的编号等等。

\subsection{已观测天区数据 obsed/}

\paragraph{runcode} 原则上每个月是一个run,以年+月(yyyymm)进行编号。对于一些特殊情况,根据以下
原则:\\
如果月底的观测延续到次月初,那么次月初的观测也属于上一个run,使用之前的编号。例如20160201就属于
201601。\\
如果一个自然月内有多个run,并且时间间隔较大,那么分别起名,例如201603A、201603B等。

\paragraph{skipped.lst} 本文件列出全部被跳过的天区编号。目前该文件列出了 Dec < 20.0 的
天区编号。跳过的天区不会被观测,在生成的各种图中也会被标注为跳过。

以下的文件均在 obsed/runcode/ 目录内。

\paragraph{files.yyyymmdd.lst} 文件列表,利用ls命令生成,每行为一个文件,记录
文件名的全名(绝对或者相对路径均可),列表中不区分是本底、平场、目标等。

\paragraph{check.yyyymmdd.lst} 检查后列表,使用 z\_Check 模块对files.yyyymmdd.lst中
列出的fits文件进行检查后生成,只保留目标观测文件信息,并且将目标名称、坐标、曝光参数等提取出来。

\paragraph{obsed.lst} 每个run目录下有且仅有一个观测结果列表,该文件使用 z\_Collect 模块
将多个 check.yyyymmdd.lst 内容汇总而成,表达本run中每个视场的观测情况,第一列为天区名,后面
依次为每个观测计划的完成率。

对于与exp\_factor.txt中记载情况都不匹配的曝光(例如一些测试观测),不会被本文件所采纳。但是本
模块不检查天区名是否在field.txt中,所以测试观测也会被列出。

\section{操作}

\subsection{策略生成}

命令: z\_Planner, yyyy, mm, dd, `hh:mm', `hh:mm' {[}, `runcode'{]} \$ \\
{[},/silent{]} {[},/overwrite{]} {[},/backup{]} {[},/simulate{]} {[},moonanglelimit={]}

基本参数说明:\\
yyyy mm dd 观测日期,年月日。\\
`hh:mm',以本地时间(根据配置中的时区)表示的观测开始时间和观测结束时间,如果时间格式不对,
或者未提供时间,那么默认为从当晚的18:00到次日06:00。\\
runcode,如果当晚的runcode不是前面给出的年月构成时,需要提供本参数,否则默认为yyyymm。

附加参数说明:\\
backup和overwrite:如果制定日期的观测计划已经存在(目录存在)那么会提示是否覆盖,或者备份原
方案,或者退出。如果指定了这两个参数中的某一个,那么执行相应的处理。\\
silent:安静模式,减少输出。\\
simulate:生成模拟观测结果,会在obsed目录中生成当晚的模拟观测结果。用于连续生成多日的观测计划。\\
moonangltlimit=:设置目标的月亮距离限制。如果未设置该项目,则自动根据当晚月相比例,选择一个
安全距离。安全距离根据月相,从10度到70度。另外,距离太阳90度内的天区也会全部被跳过。\\
距离太阳的限制,主要是为了将当晚不可能观测的天区尽可能剔除,提高程序效率。事实上不影响策略。

\subsection{策略生成过程}

\paragraph{1 读取配置} 读取现有配置文件,初始化系统。

\paragraph{2 读取已观测数据} 读入已观测列表,并且汇总每个天区的观测情况,根据完成率,计算每个天区
需要的观测次数。注意,每个天区在不同观测计划下的次数是不同的,也就是说,有些天区可能存在部分观测的
情况。

所有未完成的天区都会被保留下来,只要任意一个计划未完成,都算。

\paragraph{3 日、月影响处理} 对太阳的影响,剔除距离太阳小于90度的天区。实际上这部分天区在正常情况下
不可能被观测到,要么在日落后的西方,或者日出前的东方,都属于俯仰角较小的天区。

月球影响处理,根据月相比例,给出10-70度的距离限制,无月夜是10度,满月是70度。根据月相进行距离设置,
可以在非满月的晚上尽可能获取较大的可观测天区。也可以直接通过参数制定距离限制。

由于在一个晚上月球在天球上运行距离都不多,目标和月球距离的变动基本上在$\pm 3 ^{\circ}$之内,所以
以午夜的时候月球位置和月相为基准进行计算。

\paragraph{4 区块(条带)合并} 为方便策略生成以及分析,根据事先生成的区块分配方案,抽取出待观测天区
的区块号,估算每个区块的中心坐标,并且以该中心坐标进行观测大气质量估算。

\paragraph{5 估算``当前''时间} 以指定的观测开始时间为当前时间,计算相应JD。

\paragraph{6 计算各区块的俯仰角和大气质量} 计算各个区块在当前时间的俯仰角,以及大气质量。大气质量
直接根据俯仰角进行估算。

\paragraph{7 选取观测区块} 通过单独的函数 zj\_chooseblock 选择观测区块,目前的原则是控制俯仰角
必须大于60度,相当于大气质量小于1.15。对于地平式望远镜,还应增加俯仰角不得大于80度的限制。然后在满足
条件的天区中,根据目前策略,从低纬度向高纬度前进,所以选择尽可能低的纬度。在同纬度的区块中,选择大气质量
最小(俯仰角最大)的区块。

针对不同的观测要求,可改写 zj\_chooseblock ,不需要改变其他模块。

如果没有合适的可观测区块,那么会退出整个程序。出现这种情况,可能的原因包括:\\
观测到后期,该时刻已经没有合适的天区,此时应在观测时间分配上进行调整,或者调整策略;\\
由于月球影响,该时刻没有合适的天区,尤其是月相较大时;\\
如果直观检查认为前半夜没有合适的天区,但是后半夜可能有,那么需要人工重新选定后半夜开始观测
时间,并重新生成。否则系统不会自动进行时间递增去寻找合适的时间。

\paragraph{8 生成列表} 生成选定区块的观测列表,并且估算该区块观测所需要的时间。必要时生成该区块
的模拟观测记录。同时会生成选择区块的天区图。图中用不同颜色代表不同的大气质量,并且标注出最终选定的
区块。图中用+表示在模拟的观测时间内在地平下的天区。

对于不同望远镜控制系统,应修改 zj\_obsline 以及 zj\_radec2str 中的输出格式。

\paragraph{9 模拟时间前进}根据估计的观测时间,估算出该区块结束的时间。如果该区块结束时,尚未到
结束观测时间,则跳转到第6步,继续生成。

\paragraph{10 汇总以及格式转换} 将当晚生成的每个区块的小文件合并成大文件,并且转换格式,生成报表
以及当晚观测方案图。对模拟观测情况进行汇总。

生成过程中必要的输出,都会出现在summary.yyyymmdd.txt中,供事后检查。

\subsection{完成情况检查}

\paragraph{ls} 每天晚上观测结束后,应执行 ls 命令,将当晚所有观测文件列出到 obsed/runcode/files.yyyymmdd.lst 
文件中,以便进行后续检查。

\paragraph{Check} 完成 ls 后,执行命令: z\_Check, yyyy, mm, dd {[}, `runcode'{]} ,自动进行文件检查。
如果文件不在本地,那么建议在执行这两个步骤后,将完成的 check.yyyymmdd.lst 从服务器下载到本地。

如果需要将某些文件标注为无效,例如观测质量不好,观测出错等,可以在制定文件信息的最后一列,将1改为0,
该列用于表示文件质量。

\paragraph{Collect} 执行 z\_Collect, `runcode' ,汇总本次所有已经观测的情况。

以上步骤每个晚上观测后,或者下一个晚上生成列表之前,必须执行。

%\end{CJK*}
\end{document}

